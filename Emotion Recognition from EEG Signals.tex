\documentclass{article}
\usepackage{graphicx} % Required for inserting images

\title{Emotion Recognition from EEG Signals}
\author{Adeeba Rafi}
\date{October 2025}

\begin{document}

\maketitle

\section{Overview}
This project shows how brain signals can be used to recognize human emotions using machine learning. 
It uses EEG (electroencephalogram) data to predict whether a person is feeling \textbf{Positive}, \textbf{Neutral}, or \textbf{Negative}. The goal is to demonstrate how a computer model can learn patterns from brainwave signals and link them to emotional states.

\section{Purpose}
The purpose of this project is to explore how artificial intelligence can interpret EEG data for emotion recognition. 
It also helps understand the connection between neuroscience and machine learning.

\section{Why I Did This}
I wanted to learn how brain-computer interfaces work and how machines can understand human brain signals. 
This project helped me practice working with real EEG data, build a machine learning model, and visualize the results.

\section{How It Works}
\begin{enumerate}
    \item \textbf{Load Data:} Real EEG brainwave data is used from a public dataset.
    \item \textbf{Clean Data:} Missing values are removed to prepare clean data for training.
    \item \textbf{Train Model:} A Random Forest Classifier is used to learn emotional patterns.
    \item \textbf{Predict Emotions:} The model predicts emotions based on new EEG readings.
    \item \textbf{Visualize Results:} Accuracy and the most important EEG features are shown in graphs.
\end{enumerate}
\section{Tools Used}
\begin{itemize}
    \item Python
    \item Google Colab
    \item Pandas
    \item Scikit-learn
    \item Matplotlib
\end{itemize}

\section{Results}
The model achieved about \textbf{96.8\% accuracy}, which means it correctly predicted emotions in almost all test cases. 
It also showed which EEG signal patterns had the biggest influence on the predictions.

\section{What I Learned}
\begin{itemize}
    \item How EEG data represents brain activity and emotions
    \item How to process and analyze real-world data
    \item How Random Forest models make predictions
    \item How AI can be applied to emotion recognition and neuroscience research
\end{itemize}

\section{How to Run}
\begin{enumerate}
    \item Open the notebook in Google Colab
    \item Upload the EEG dataset file (\texttt{data.csv}) from the EEG Brainwave Dataset: Feeling Emotions on Kaggle
    \item Run each code cell step by step
    \item View the accuracy, predictions, and visualizations
\end{enumerate}

\section{Dataset}
\textbf{Name:} EEG Brainwave Dataset: Feeling Emotions \\
\textbf{Source:} Available publicly on Kaggle

\section{Summary}
This project shows how brainwave data can be used to identify emotional states using machine learning. 
It is a simple example of how computers can interpret human emotions from EEG data.

\section{Future Improvements}
\begin{itemize}
    \item Add real-time emotion detection using live EEG signals
    \item Create a visual interface to display predicted emotions
    \item Experiment with deep learning models for higher accuracy
\end{itemize}
\end{document}
